Як відомо, ізольовані кварків або глюони ніколи не спостерігалися. Спостерігалися лише їх зв’язані стани, відомі як частинки адрони, які розділені на дві групи: мезони і баріони. Динаміка кварків в складі мезонів і баріонів описується квантовою хромодинамікою, яка не є пертурбативною на енергетичній шкалі адрона. Тому для їх опису було розроблено кілька теоретичних методів та моделей. Сюди входять зокрема кваркова модель,  КХД на гратках, правило сум КХД, тощо. Незважаючи на відносно просту форму, кваркова модель досить добре відтворює спектри експериментально спостережуваних адронів. Тому кваркова модель на сучасному етапі досліджень покладена в основу нашого розуміння і опису властивостей  адронів. Однак зв'язок між КХД та кварковою моделлю досі залишається не описаним із перших принципів. Розрізняють два типи кваркових систем: 

1) кварк-антикваркові структури, в яких кварк ніби закритий для спостереження своїм антикварком (hidden) 

2) так звані відкриті (open) кваркові структури, в яких комбінують різні типи кварків, зокрема важкі і легкі кварки. 

Для кращого розуміння динаміки кварків всередині адронів хорошим дослідницьким зондом є саме важкі адрони з відкритим ароматом, які містять один важкий c або b кварк у сполученні із одним або кількома легкими кварками.

Відкриті чарівні адрони також мають важливу експериментально спостережувану особливість -- вони мають меншу ширину в порівнянні з адронами, які не містять важких кварків. Наприклад, $\rho$-мезон, який характеризується станом $J^P = 1^{-}$ і складається з кварків u і d, має ширину більше 100 МеВ, тоді як відповідний $1^{-}$ стан мезона $D^{\star}$ має ширину менше 1 МеВ. Якщо $\Lambda(1405)$ $\Lambda(1520)$ баріони в станах $1/2^{-}$ і $3/2{-}$ стану мають ширину приблизно 50 МеВ та 15 МеВ відповідно, то відповідні стани у секторі чарівних $\Lambda_{с}$c баріонів із масами 2593 МеВ та 2625 МеВ  мають ширину близько 2.5 МеВ та <1.0 МеВ відповідно. Завдяки вузькій ширині експериментально легко виявити чарівні адрони і виміряти їх різні властивості.
